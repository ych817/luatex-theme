\usemintedstyle{material}                                    % 代码块主体高亮风格(见https://pygments.org/styles/)
\colorlet{colorCodefenceBodyDraw}{Gray}                      % 代码块主体边框颜色
\colorlet{colorCodefenceBodyFill}{Black!75!DarkBlue}         % 代码块主体填充颜色
\colorlet{colorCodefenceBodyFillHL}{black!45!DarkSlateBlue}  % 代码块主体填充颜色-高亮行

\colorlet{colorCodeinlineBodyDraw}{Gray}                     % 行内代码块边框颜色
\colorlet{colorCodeinlineBodyFill}{Black!75!DarkBlue}        % 行内代码块填充颜色
\colorlet{colorCodeinlineBodyFillHL}{black!45!DarkSlateBlue} % 行内代码块填充颜色-高亮行

\colorlet{colorCodefenceNmbrText}{white!25!LightSteelBlue}   % 代码块行号文字颜色
\colorlet{colorCodefenceNmbrFill}{colorCodefenceBodyFill}    % 代码块行号填充颜色

\colorlet{colorCodefenceTitlText}{Cerulean!50!white}         % 代码块行号文字颜色
\colorlet{colorCodefenceTitlDraw}{colorCodefenceNmbrText}    % 代码块标题边框颜色
\colorlet{colorCodefenceTitlFill}{colorCodefenceBodyFill}    % 代码块标题边框颜色

\tcbset{
  codestyle/.style={
    breakable,
    enhanced,
    skin first=enhanced,
    skin middle=enhanced,
    skin last=enhanced,
    colframe=colorCodefenceBodyDraw,
    colback =colorCodefenceBodyFill,
    fontupper=\footnotesize,
    boxrule=.4pt,
    arc=5pt,
  }
}
\setminted{
  breaklines=true,
  breakanywhere=true,
  escapeinside= 脎吡,%
  encoding=utf8,
}

% 代码块中高亮行的行号也能被高亮 。代码来自 https://tex.stackexchange.com/questions/741923
% 相关的库 : minted , fbextra , fancyvrb , etoolbox
\makeatletter
\RenewDocumentCommand{\theFancyVerbLine}{}{%
  \ttfamily\color{colorCodefenceNmbrText}%
  \ifx\FV@LeftListNumber\relax
  \else
    % if left line number is one
    \ifcsname FV@HighlightLine:\arabic{FancyVerbLine}\endcsname
      % and if current line is highlighted
      \setlength{\fboxsep}{0pt}%
      \colorbox{\FancyVerbHighlightColor}{%
        \hspace{\myFV@EstimatedNumberWidth}%
        \strut % \strut in the default font size
        \footnotesize
        \vphantom{000}\llap{\oldstylenums{\arabic{FancyVerbLine}}}%
        \hspace{\dimexpr+\FV@NumberSep}}%
      \hspace{-\FV@NumberSep}%
    \else
      \footnotesize
      \oldstylenums{\arabic{FancyVerbLine}}%
    \fi
  \fi
}
\newlength{\myFV@EstimatedNumberWidth}
\setlength{\myFV@EstimatedNumberWidth}{18pt}
\makeatother
% 代码块外观样式设置
% 相关的库 : minted , tcolorbox , xparse
\NewTCBListing{codefence}{ !o !E{\Syntax\Highlight\Insert\Delete}{{latex}{}{}{}} }{%
  codestyle,
  boxsep=2pt,
  right=-2pt, % 对应 boxsep
  overlay={
    % 下面的命令用于在代码块中使文件名和语法可以不被高亮色块遮挡
    \pgfdeclarelayer{background}               % 
    \pgfdeclarelayer{foreground}
    \pgfsetlayers{background,main,foreground}
    \begin{tcbclipframe}
      \tikzset{
        boxes/.style={
          draw=colorCodefenceTitlDraw,
          % fill=colorCodefenceTitlFill,
          text=colorCodefenceTitlText,
          line width=.4pt,
          font=\ttfamily\footnotesize,
          rounded corners=3pt,
          outer sep=2pt
        }
      }
      \begin{pgfonlayer}{foreground}
      \node(ext) [anchor=north east,boxes] at (frame.north east) {#2};
      \IfNoValueF{#1}{
        \node(title) [anchor=east,boxes] at (ext.west) {#1};
      }
      \end{pgfonlayer}
    \end{tcbclipframe}
  },
  listing only, 
  listing engine=minted,
  minted language = #2, 
  minted options = {   % minted 设置
    tabsize=2,
    linenos=true,
    numbersep=5pt,
    highlightlines={#3},
    highlightcolor=colorCodefenceBodyFillHL,
    xleftmargin=+10pt,
  }
}
% 行内代码块外观设置
% 相关的库 : tcolorbox
\NewTotalTCBox{\codeinline}{ s E{\Syntax}{{latex}} m }{% 行间代码块设置
  codestyle,
  boxsep=0pt,
  left=2pt,
  right=2pt,
  top=2pt,
  bottom=2pt,
  on line, % 位置设置
}{%
  \IfBooleanT{#1}{% 如果命令尾部有星号 %
    \textcolor{gray}{% 就补上灰色的
      \ttfamily >\ % "Prelude>"
    }%
  }%
  \vphantom{\tt (Fg)}\smash{\mintinline[%
    %
  ]{#2}{#3}}%
}

% 代码块中高亮一行内的部分内容 。
% 相关的库 : 
\newcommand{\codeHL}[2][colorCodefenceBodyFillHL]{%        % 代码高亮部分
    \setlength\fboxsep{0pt}%
    \rlap{\colorbox{#1}{\vphantom{(Fg)}\phantom{#2}}}%
    \setlength\fboxsep{3pt}%
}%